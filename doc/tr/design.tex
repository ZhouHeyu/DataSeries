\section{Design}\label{sec:design}

DataSeries is intended to provide streaming access to structured serial
data. Corresponding
to the first four properties\footnote{The fifth property, an expressive
programming interface, is described in Section~\ref{sec:programming}.}
described in the introduction, DataSeries was designed with the
following goals in mind. First, it should be very storage
efficient. Second, it must be efficient to encode, decode and
interpret the data.  Third, the format should not constrain the
types of information to be stored. Fourth, the internal data must be
self-describing, i.e., the names and types of the data stored have to
be determined by the contents of the file itself, rather than 
externally.

The file structure and type definitions are described in the user guide.
% TODO: do the cross-file references stuff; slightly annoying since it's cross directory.
% something like:
% \externaldocument[userguide-]{dataseries-userguide}
% ADD_DEPENDENCIES(latex_tr latex_userguide)

\subsection{Extent types and options}\label{sec:extenttype}

Extent types and options are now described at
$<$https://github.com/dataseries/DataSeries/wiki/Defining-types$>$.

\subsection{Design summary}

The DataSeries file format was designed to allow for flexibility
(through the use of a self-contained and extensible type description
for extents) and performance (through extensive use of compression and
a data layout that allows for direct access to data values). 
Section~\ref{sec:results} describes experiments using
DataSeries that quantitatively validate these claims.


% TODO: add something about time formats in here; include http://cr.yp.to/libtai/tai64.html

