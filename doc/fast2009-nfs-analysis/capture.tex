\section{Capture}
\label{sec:capture}

The first stage in analyzing an NFS workload is capturing the data.
There are three places that the workload could be captured: the
client, the server, or the network.  Capturing the workload on the
clients is very parallel, but is difficult to configure and can
interfere with the real workload.  Capturing the workload on the
server is straightforward if the server supports capture, 
but impacts the performance of the
server.  Capturing the workload on the network through port mirroring
is about as convenient as capture on the server, and given that most
switches implement mirroring in hardware, has no impact on network or
workload performance.  On fiber networks, even the potential impact of
port mirroring can be eliminated through the use of optical fiber
splitters. Therefore, we have always chosen to capture the
data through the use of port mirroring.

A single mirror port could be insufficient for mirroring both the send
and recieve directions of a full-duplex port.  However, advanced
switches can selectively mirror each direction of a port onto
particular ports.  We used this functionality in our 2003 tracing to
spread 2-4 one Gbit links over 2 one Gbit mirror ports.

Even if 1 second average rates are low enough to fit onto the mirror
ports, if the switch has insufficient buffering, packets can still be
dropped. We discovered this problem on a switch that used per-port
rather than per-card buffering.  To eliminate the problem, we switched
to 10Gbit mirror ports to reduce the need for switch-side buffering.

The capture host can also be overrun. At low data rates (900Mbits,
70kpps), standard tcpdump on commodity hardware works fine.
However, at high data rates (5000Mbits, 1,000kpps),
traditional approaches are insufficient. Indeed,
Leung\cite{LeungUsenix08} notes
% pg 215 ``when tcpdump dropped a few packets''
difficulties with packet loss using tcpdump on a 1 Gbit mirror port.
We have developed three separate techniques for packet capture, all of
which work better than tcpdump: {\it lindump}(user-kernel ring
buffer), {\it driverdump}(in-kernel capture to files), and {\it
endacedump}(hardware capture to memory).  Further details and
experiments with the first two techniques can be found
in~\cite{Anderson06network-tracing}.

\subsection{Lindump}

The linux kernel includes a memory-mapped, shared ring buffer for
packet capture.  We modified the example lindump program to write out
pcap files~\cite{pcap}, the standard output format from tcpdump, and
to be able to capture from more than one interface at the same time.
We wrote the output files to an in-memory filesystem using mmap to
reduce copies, and copied and compressed the files in parallel to
disk.  Using an HP DL580G2, a current 4 socket server circa 2003,
lindump was able to capture about 3x the packets per second (pps) as
tcpdump and about 1.25x the bandwidth.  Combined with a somewhat
higher burst rate while the kernel and network card buffered data,
this was sufficient for mostly loss free captures at the animation
company, and was the technique we used for all of the 2003 set of
traces.

Once our packets are captured into files in tmpfs, we used compression
to maximize the effective disk space.  If the capture host is mostly
idle, we compressed with {\tt gzip -9}. As the backlog of pending
files increased, we reduced the compression algorithm to {\tt gzip
-6}, then to {\tt gzip -1}, and finally to nothing.  In practice this
approach increased the effective disk size by 1.5-2.5x in our
experience as the data was somewhat compressible, but at higher input
rates we had to fall back to reduced compression.

\subsection{Driverdump}

At a second site, our 1Gbit lindump approach was insufficient because
of packet bursts and limited buffering on the switch.  Replacing the
dual 1Gbit cards with a 10GbE card merely moved the bottleneck to the
host and the packets were dropped on the card before they could be
consumed by the kernel.

To fix this problem, we modified the network driver so that instead of
passing packets up the network stack, it would just copy the packets
in pcap format to a file. Then it would immediately return the packet
buffer to the NIC.  A user space program prepared files for capture,
and closed the files on completion.  We called our solution {\tt
driverdump} since it performed all of the packet dumping in the
driver.

Because driverdump avoids the kernel IP stack, it could
capture packets faster than the IP stack could drop them.
% measured by using tcpdump with a filter that drops all packets. 
We increased the sustained packets per second over lindump by 2.25x to
676,000pps, and sustained bandwidth by 1.5x to 170MiB/s.  We could
handle short bursts up to 900,000 pps, and 215 MiB/s.  This gave us
nearly lossless capture to memory at the second site.  Since the files
were written into tmpfs, we re-used our technology for compressing and
copying the files out to disk.

\subsection{Endacedump}

In 2007, we returned to the animation company to collect new traces on
their faster NFS servers and 10GbE network.  While an update of
driverdump might have been sufficient, we decided to also try the
Endace DAG 8.2X capture card~\cite{endace-cards}.  This card copies
and timestamps packets from a 10GbE network directly into memory.  As
a result, it can capture minimal size packets at full bandwidth, and
is intended for doing in-memory analysis of networks.  Our challenge
was to get the capture out to disk, which was not believed to be
feasable by our technical contacts at Endace.

To solve this problem, we integrated our adaptive compression
technique into a specialized capture program, and added the
lzf~\cite{lzf} compression algorithm, that compresses at about
100MB/s.  We also upgraded our hardware to an HP DL585g2 with 4 dual
core 2.8GhZ Opterons, and 6 14 disk SCSI trays.  Our compression
techniques turned our 20TiB of disk space into 30TiB of effective disk
space.  We experienced a very small number of packet drops because our
capture card limited a single stream to PCI-X bandwidth (8Gbps), and
required partitioning into two streams to capture 10GbE.  Newer cards
capture 10GbE in a single stream.

\subsection{Discussion}

Our capture techniques are directly applicable to anyone attempting to
capture data from a networked storage service such as NFS, CIFS, or
iSCSI.  Our most advanced techniques are capable of lossless
full-packet capture at 10GbE, but require the use of special hardware.
Our simplest technique allows capture at over twice the rate of
tcpdump, and expands the effective size of the disks by 1.5x through
adaptive compression. Our intermediate technique increases the capture
rates by a factor of 2-3x at the cost of development time.  Both the
lindump and driverdump code are available in our source
distribution~\cite{DSOpenSource}.  These tools and techniques should
eliminate problems of packet drops for capturing storage traces.
