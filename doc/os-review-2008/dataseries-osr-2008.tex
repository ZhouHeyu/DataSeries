\documentclass{acm_proc_article-sp}
\makeatletter
\let\@copyrightspace\relax
\makeatother

% \newcommand{\fix}[1]{\marginpar{\LARGE\ensuremath{\bullet}}\textbf{[#1]}}
\newcommand{\fix}[1]{{\LARGE\ensuremath{\bullet}}\textbf{#1}}

\begin{document}
% replace the name to unblind
\newcommand{\DataSeries}{DataSeries}
\newcommand{\DS}{DS}

\title{DataSeries: an efficient, flexible data format for structured serial data}
% Pretend we have one author, minimizes the space we waste on that.
\numberofauthors{1} 
\author{
\alignauthor
Eric Anderson, Martin Arlitt, Brad Morrey, Alistair Veitch  \\
 \affaddr{HP Labs. 1501 Page Mill Rd.  Palo Alto, CA} \\
 \email{\{eric.anderson4, martin.arlitt, brad.morrey, alistair.veitch\}@hp.com}
}

\maketitle
% % A category with the (minimum) three required fields
% \category{H.4}{Information Systems Applications}{Miscellaneous}
% %A category including the fourth, optional field follows...
% \category{D.2.8}{Software Engineering}{Metrics}[complexity measures, performance measures]
\category{H.3.2}{Information Storage}{}[]
 
% \terms{Structured serial data}
\terms{Design, Performance}

% \keywords{ACM proceedings, \LaTeX, text tagging} % NOT required for Proceedings
\keywords{data format, compression, performance}

\begin{abstract}
\textit{Structured serial data} is used in many scientific fields;
such data sets consist of a series of records, and are typically
written once, read many times, chronologically ordered, and read
sequentially.
In this paper we introduce DataSeries, an on-disk format,
run-time library and set of tools for storing and analyzing 
structured serial data.
We identify six key properties of a system to store and analyze
this type of data, and describe how DataSeries was designed
to provide these properties.  We quantify the benefits of DataSeries
through several experiments.  In particular, we demonstrate
that DataSeries exceeds the performance of common trace formats
by at least a factor of two.
\end{abstract}

\section{Introduction}\label{sec:intro}

Traces, recordings and measurements taken from computer systems,
networks and scientific infrastructure are vitally important for a
large variety of tasks. In every area of computer system design,
traces from existing systems have been used to validate hypotheses,
test assumptions and estimate performance. This is true of I/O
subsystems~\cite{IORef,Ji03,Uysal03}, processor
systems~\cite{ProcRef}, network systems~\cite{NetRef} and memory
systems~\cite{MemRef}, among others. Traces and logs are also
extremely useful for fault-finding, auditing and debugging purposes
~\cite{DebugRef}. Traces composed of failure data have been used to
determine system reliability~\cite{ReliabilityRef, Schroeder07,
Pinheiro07}. Trend analyses of performance information is a core
operation of various management tools~\cite{MgmtRef}. Scientific and
medical instrumentation can also generate large amounts of
data~\cite{SciRef}, which also needs to be stored, filtered and
analyzed.

The data stored in each of these diverse uses is {\it structured
serial data}, which we define as a series of records, each record
having a specified structure (i.e., containing the same set of
variables or fields). Structured serial data has four defining characteristics:
its structure is record-oriented; it is typically written only once,
and is read many times; it is usually ordered
in some manner, e.g., chronologically; and it is typically read
sequentially.  We have designed and built DataSeries, an on-disk 
data format, run-time library, and set of
tools that is optimized for storing and analyzing this type of data.
We show that 
%the performance of 
DataSeries outperforms
%exceeds the performance of 
common trace formats and databases by at
least a factor of two, and in some cases up to an order of
magnitude. DataSeries also requires far less disk space (factors vary
from 4x to 8x in test data sets).

% which we define as an ordered series of records that share a common
% structure.  This type of data commonly occurs as trace data in
% computer systems, but since the format is essentially ordered RDBMS
% tables, the need to maintain and analyze such data occurs in a large
% number of scientific fields.

We desire six key properties of a data format and analysis system for
structured serial data:

\begin{enumerate}

\item \textbf{Storage efficiency}: the data should be stored in as few
bytes as possible. There are several driving factors behind this
requirement. Firstly, the amount of data stored can be large (we have
I/O traces comprising billions of records), and despite rapidly
decreasing storage prices, the cost to store data can still be
considerable, particularly when the data must be kept for long periods
of time. Secondly, we have learned that one of the primary factors
behind analysis efficiency is the speed at which data can be
retrieved. Regardless of the storage technology used, more highly
compressed data can substantially speed access times.

\item \textbf{Access efficiency}: accessing, interpreting and encoding
trace data, whether reading or writing, should make efficient use of
CPU and memory resources. From experience and experimental analysis,
we have learned that the second major factor determining analysis
efficiency is the CPU overheads of interpreting data once it has been
read off disk.

\item \textbf{Flexibility}: adding additional fields should not affect
users of the trace data.  Removing or modifying data fields should
only affect programs that use those fields, and the system should
catch incorrect usage.  Further, the format should not constrain
the type of data being stored, and should allow multiple record types
in a single file. 
%Again, 
Experience 
%has taught 
tells us that formats that
are not flexible lead to severe maintenance issues for both the format
interpretation code and analysis systems.

\item \textbf{Self-describing}: the data set should contain the
metadata that describes the data. This is another experience-driven
requirement, as we have had problems in trying to interpret and use
trace data from other organizations, and maintaining organizational
knowledge of our own metadata over time.

\item \textbf{Usability}: the data format should have an
associated programming interface that is both expressive and easy to
use. 
%In particular 
The user model of the data and its analysis should
be easy to describe and the interface to it should easily allow for
common operations (e.g., scanning an entire data set, and processing
specific fields only).

\item \textbf{Integrity}: many structured serial data files are 
intended as archival traces. To protect against media
errors or incorrect software systems, DataSeries files need to be
self-contained and contain internal checksums that enable
integrity checking.

\end{enumerate}

There are, inevitably, tradeoffs to be made in these requirements. For
instance, XML is an extremely flexible data format, but is not very
efficient. In the course of our work (which tends towards the analysis
of very large data sets), and the design of DataSeries, we have chosen
to prioritize efficiency over many of the other properties.

Although numerous tracing and measurement systems have been developed
over the last 20-30 years, we are not aware of any that meet all of
these requirements. We analyze some of these in our description of
related work (Section~\ref{sec:related}).

We provide four primary contributions in this paper.  First, we
introduce DataSeries, a data format and associated library, which was
specifically designed to meet the five key properties discussed above,
and relate some of the experiences that led us to make various
design decisions. 
Second, we discuss how DataSeries can support very large data sets
(e.g., hundreds of billions of records) on modest systems.  Third, we
describe how we have used DataSeries in practice to store a wide
variety of data types.  Fourth, we demonstrate the performance and
storage efficiency of DataSeries in a set of controlled experiments,
using empirical data sets. Throughout, we have tried to emphasize the
lessons learned and how they might be applied to other fields.

DataSeries software is publicly available under a BSD license from \texttt{http://tesla.hpl.hp.com/opensource/}. Given the benefits of DataSeries that we demonstrate, we
argue that DataSeries should be considered for use by any application
that needs to store large amounts of structured serial data. Indeed
the Storage Networking Industry Association (SNIA) I/O traces tools
and analysis (IOTTA) technical working group~\cite{iotta-website} has
proposed DataSeries as the standard format for I/O trace data.
% and is
% currently specifying the semantics for I/O traces encoded using DataSeries.

The remainder of this paper is organized as follows.
Section~\ref{sec:related} describes the strengths and weaknesses of
existing storage technologies relative to DataSeries.
Section~\ref{sec:design} describes the design of DataSeries, including
on-disk and in-memory formats, and introduces the programming model.
Section~\ref{sec:results} presents empirical and benchmark results
%from our use of DataSeries 
to illustrate and quantify the benefits of
DataSeries. Section~\ref{sec:conclusions} concludes the
paper with a summary of our work.
% and a list of future directions.

\section{Related work}\label{sec:related}

We classify the related work into three categories:
those that use a customized binary format, those that use a
text-based format, and relational database systems. 
For more complete and quantitative comparisons,
% including performance results, 
please refer to~\cite{DSTechnicalReportSnapshot}.

Custom binary formats are usually serialized or directly written
versions of an in-memory data structure.  As such, they usually
achieve storage 
%efficiency 
and access efficiency, 
but fail to a\-chieve
flexibility, self-describing, usability, and integrity.
However, as we
show in Section~\ref{sec:efficiency}, 
unless the authors are careful they can also
fail to achieve access efficiency.

Text formats such as Column Separated Value (CSV) can
%often (but far from always)
achieve flexibility and self-describing. XML achieves flexibility,
self-describing, usability, and mostly integrity.  However, they fail
to achieve storage efficiency, and can fail access efficiency by
multiple orders of magnitude.  Even very tuned CSV implementations can
only get to within 2-7x the access-efficiency of DataSeries, and 4-7x
for the storage efficiency.
% (See the DataSeries technical 
%report~\cite{DSTechnicalReportSnapshot} for these results.)
%{\bf TODO: need to re-do these experiments at least to
%measure the file sizes, and potentially with Tfrac text files,
%although I suspect people would use sec.usec in a block I/O trace}

Relational databases achieve properties flexibility, self-de\-scrib\-ing,
usability, and integrity. RDBMS's were designed to handle updates, so
do very limited compression drastically hurting their storage
efficiency.  Our
results show $>$10x improvement on storage efficiency for DataSeries
over MySQL.
%{\bf TODO: check the exact sizes, should be around there.}
Similarly, the generality of SQL can hurt it.  Even for fairly simple
queries running entirely on in-memory data, DataSeries runs 2-7x
faster than MySQL. 
%{\bf TODO: re-do these numbers with parallel
%decompression, etc.}  
Retrieving the data for a more complicated
calculation on the client would further slow the relative performance.

\section{Design}\label{sec:design}

DataSeries' data model is conceptually very similar to that used by
relational databases.  Logically, a DataSeries file is composed of an
ordered sequence of {\it records}, where each record is composed from
a set of {\it fields}. Each field has a {\it field-type} (e.g.,
integer, string, double, boolean) and a name. A DataSeries record is
analogous to a row in a conventional relational database. For
efficiency, we group a set of rows that have the same fields and field
types into an extent.  We call the type of that extent the {\it
extent-type}.  Usually an analysis 
%will be run 
runs over a collection of
extents, so a collection of extents of the same type is similar to a
database table.

\subsection{Data format}

A single DataSeries file comprises one or more extents
(potentially with different extent-types), a header and
extent-type extent at the beginning of the file, and an index extent
and trailer at the end of the file. The header on a DataSeries file 
contains the DataSeries
file version, and check values that enable the reader to determine the
endianness encodings of the data types.  DataSeries files are always
written out using the native formats of the writing system. This 
typically minimizes byte-swapping overheads, as the architecture 
reading the files is almost always the same as the architecture reading
the files. The library transparently converts in the rare cases this
is not true, and DataSeries files can be explicitly ``repacked'' if desired. 

The extent-type extent contains records with a single string-valued
field, each of which contains an XML specification that defines the
extent-types of all the other extents in the file. We chose to use XML
%to define our extent types 
%because 
as we saw no point in creating a new
grammar 
%for representing 
to represent this information, it is flexible (for
instance, it easily allows the addition of options describing
fields) and allows embedded comments.
%% and SQL create table
%% statements did not have a representation for the packing options we
%% wanted to specify.

The trailer
consists of the offset and size (after compression) of the index
extent.  The offset is used to read the index extent, which has two
fields, an extent-type and an offset, to allow direct access to
extents of a single type. Storing the index and trailer at the end
of the file enable efficient writing, since writing applications will
often not know
the final extent sizes \textit{a priori},
%, a priori, what the final extent sizes will be, 
so space 
for the index cannot be allocated until after the extents have been written.

Each extent consists of a header, followed by the fixed sized fields
and separately the variable sized data.  A reference to the variable
sized data from the fixed size fields allows for variable sized
fields.  This separation allows for direct access to fixed sized
fields since they will be at a known offset, and one indirection to
get at the variable sized data.  Both fixed and variable sized data
may be compressed, using any one of a number of standard compression
algorithms~\cite{BZIP,GZIP,LZF,LZO}.  The extent header contains
metadata about the data in the extent, such as the compressed sizes of
the fixed and variable data, the number of records in the extent, the
uncompressed size of the variable length data, the compression mode,
the extent-type of the extent, and checksums of the extent before and
after compression to guard against hardware and software errors.
Programs usually disable checksum validation during extent reading to
improve performance at the cost of reduced integrity.

The extent format is designed for efficient access. Values are packed
so that once an extent is read into memory, an analysis can iterate
over the rows simply by increasing a single counter, as if for an
array containing structures.  Individual values are accessed by an
offset from that counter and a C++ cast operation. 
%This technique
%could be applicable 
%to anyone storing similar data, e.g., RPC fields or
%a streaming RDBMS.

As we initially built DataSeries prototypes, we realized the need to
add options that control various aspects of the field and extent
descriptions in order to meet the storage and access efficiency and
flexibility requirements.  While space precludes a full description
(see~\cite{DSTechnicalReportSnapshot}), we describe some of the more
significant below:


\begin{itemize}

\item \textbf{nullable fields}: there are many instances when a field
value may or may not be present, and we found it useful to be able to
indicate that a given value is null. This option is implemented by
generating a hidden boolean column that determines if the value is null.

\item \textbf{double scale}: often, the full precision of double
values is not needed. In this case, it is possible to get
significantly more compression by zeroing unimportant lower
order bits by scaling and rounding. This option lets the user specify
a specific scale factor for the values stored.

\item \textbf{relative packing}: specify that a given field be packed
relative to another, using delta encoding. This option is also useful
for timestamps and index values that may be large, but are small
relative to each other. This enables fewer bits to be used to store a
given value.

\item \textbf{unique packing}: ensures that every string (arbitrary
length binary data) field is stored only once. If a 
%given 
data set
contains many repeated values, this option can significantly increase
the 
%effective 
compression ratio, as compression algorithms only partially
remove duplicate data.

\item \textbf{versioning}: since our format is archival, we expect
changes in the fields used in a dataset over time.  To handle these
changes we 
%have
introduced a version number associated with a type.
A program compatible with version $n.m$ of the format will operate
properly on any file with version $n.m', m' >= m$.  Programs can also
detect the version number and adapt appropriately.

\end{itemize}

One addition we did try, and later rejected, was that of extending a
double fields precision by allowing the specification of a base value.
This was introduced when we realized that storing microsecond precision
time values, using the UNIX epoch (00:00:00 UTC on January 1, 1970),
was not possible with a single double. We added an option to
specify a base value for the double fields, and then stored the actual
value relative to that base value.
Unfortunately, experience showed that this option
was less useful than hoped for. It proved difficult and confusing for
those developing analyses to deal with, particularly in the case where
multiple files, each with different bases, were being used. Arithmetic
on values of this type was complicated and inefficient. Ultimately, we
deprecated this option, in favor of other time representations,
dependent on the domain. 

We have now chosen to explicitly represent the units and the epoch for
time values.  The units are chosen based on the original precision,
e.g., nanoseconds for NFS attribute values, or microseconds for pcap
trace files.  We then use a 64-bit integer to store the time value,
and have a special field that can translate in and out of the raw
units from input formats such as a double, or a (second, nanosecond)
pair.  Analysis are written to operate over the ``raw'' time format,
and then they use the field to convert to more normal formats for
printing.  If the analysis needs to calculate windows, it can convert
a normal format to a ``raw'' format.  For example to calculate mean
bytes per 30 second window, an analysis would first convert 30 seconds
into the raw format, it would then accumulate values for that
interval, output the mean and reset the statistic.  The use of units
and epoch also allows us to print out time values in a nicer human
readable form regardless of the input type.  To deal with older files
that are missing units and epoch, we added the ability to specify
units and epochs based on the extent type, extent version and field
names.

If there is no natural unit for a set of traces, for example because
they have cycle counter times, then we recommend units of $2^{-32}$
seconds.  This choice allows for the maximum precision possible when
providing the same range as a UNIX timevalue since the UNIX timevalue
uses 32 bits to represent the seconds.  We expect at some point to
need to move to a 128 bit fixed or floating point representation to
deal with the continually increasing range of times for trace data and
the increasing precision of the clocks measuring the times.  The
maximum required bits would be around 200 since that would be
sufficient bits to represent all times for the age of the universe
using the Plank time granularity.


\subsection{Programming model}

Four general functionalities are supported by DataSeries:
reading a DataSeries file, 
analyzing the data in a DataSeries file,
writing a DataSeries file,
and writing an alternative output format (e.g., CSV).
These functionalities meet all of the typical needs for
users of structured serial data, and thus facilitate
the usability (property 5) of DataSeries.

There are three key concepts in the C++ API provided by
DataSeries. The first is that of an \textit{ExtentSeries}, which is an
iterator over the rows in at least one extent.  Iterators in
dataseries are similar to iterators in relational
databases~\cite{graefeQueryProcessing93}, but they typically operate
in bulk over an entire extent in a single pass to improve access
efficiency.  Iterators in dataseries are also not required to operate
over extents of identical type.  An iterator specifies its
compatibility for types.  Normally compatibility is exact type
compatibility, but it can also be loose, which means that the types
are compatible if all the fields can be found.  The second key concept
is that of a \textit{module}, which accepts a series of extents
processes them and passes them to downstream modules.  Similar to
River~\cite{river99}, modules can have multiple inputs if they are
joining together two different series.  The ability to add
functionality in a modular fashion enhances the flexibility (property
3) of DataSeries.  


Analysis programs generally have a main program that builds a sequence
of modules to perform multiple analyses in a single pass over the
data. Each module processes an extent at a time, and iterates over
each of the rows in the extent.  For easier usability, a row analysis
module can be used that will call a processRow() function on each row.
The {\tt dstypes2cxx} program will automatically generate the boilerplate
needed for a module to be used as a row analysis module.
For efficiency reasons, the transfer of extents between
modules is a transfer of ownership, i.e., the module that returns an
extent must not continue to access it. We are considering relaxing
this constraint in the future to allow multiple modules to read-share
a single extent at the same time so that we can increase the 
%available
parallelism for analysis without increasing the memory requirements.
%Each module can iterate over the rows in
%the extent by putting it into a series and using the series iterator
%to select the row and the fields to access rows in that series.
DataSeries also provides modules which exploit parallelism in
operations, such as decompressing or compressing and writing extents
to disk in parallel, 
%which increases 
to increase the access efficiency on
multi-core systems.  This improvement is very important for
compression, which is usually slow, but can also be important for
simple analyses that run faster than a single core can decompress.

The DataSeries source distribution contains numerous other built-in
general modules, for example, ones that convert extents to text, and
ones that perform simple versions of the SQL select/group-by
statement.  We also have converters from various input formats (pcap,
log file, SRT~\cite{SRT}) into a DataSeries representation.  Lastly,
there are type specific analysis for converted NFS, LSF batch
jobs~\cite{PlatformLSF} and logical/physical disk volume traces.  More
detailed information on these modules and programming examples are
available 
% in the DataSeries technicalreport
in~\cite{DSTechnicalReportSnapshot} and the source distribution.

\section{Performance results}\label{sec:results}

%{\bf TODO: I wonder if we should rename these ``Harvard'' traces
%rather than Ellard?}
%{\bf TODO: add sections here to fill up space somewhat}
%We performed various experiments to measure the effectiveness of
%DataSeries' compression techniques, and then further compared
%DataSeries to data encoding and analysis using MySQL, CSV, CStore,
%Dan Ellard's NFS traces~\cite{ellard03}, and the 1998 World Cup Web
%traces~\cite{ita-wcweb98} for compression ratio and analysis execution speed.
%Due to space constraints we give example results from our
%comparison with Ellard and comment on highlights of our other results.

% added by Martin
In this section we provide examples to illustrate quantitatively the 
effectiveness of DataSeries.  
Section~\ref{sec:scale} describes how we have used DataSeries to store 
and analyze a large dataset on a modest system.
Section~\ref{sec:efficiency} demonstrates the flexibility provided
by DataSeries for selecting between storage and access efficiency needs.
Section~\ref{sec:ellard} provides a case study which illustrates
both the reusability of DataSeries and its access efficiency compared
to existing work.
Additional examples and comparisons can be found 
%in the DataSeries technical report
in~\cite{DSTechnicalReportSnapshot}.

\subsection{Scalability}\label{sec:scale}

The largest dataset we have is a trace of NFS traffic to and from
busy enterprise file servers.
The primary extent type 
in this data is the common records which store information about each
of the 200 billion request and reply messages. We have secondary tables that
store information about each packet captured, operations that included
file attributes, read and write requests and mount requests.  The total dataset
is about 5TB.

As a demonstration of the real-life performance of
DataSeries, consider the following example.  Utilizing our NFS
trace, we performed an analysis of the throughput
%effects if servers were instead accessed across a WAN.
% MFA: need to change above description of trace if we say servers
effects if the server was instead accessed across a WAN.
The analysis read in 45.5 GB of data (406 GB when uncompressed), and
processed 7.6 billion records (each record corresponds to an NFS
transaction).  Using a 2003 model two-way 2.8 GHz Xeon server, the
entire data set was processed in 11,263 wall clock seconds (about 3
hours), or roughly 675,000 rows per second while performing a set of complex
analyses. \fix{need to get current performance numbers}

\subsection{Examining storage and access efficiency}\label{sec:efficiency}

We have performed extensive experiments on the effect 
of compression
on the performance of analysis~\cite{DSTechnicalReportSnapshot}.  We
summarize the results on storage efficiency below:

\begin{itemize}

\item For archival storage, or preparation for network distribution,
bzip2~\cite{BZIP} compression with large extents (16-64MB) makes the
most sense.  Keeping the size at most 64MB allows for some parallelism
during compression, and later decompression, but extracts almost all
of the potential compression available.  It may be useful to leave on
gzip~\cite{GZIP} compression 
as gzip occasionally gets better compression that
bzip2.

\item For analyses with sufficient disk bandwidth, lzo~\cite{LZO} with
small extents (96-128KB) gets the maximum performance.  The optimal
extent size is somewhat below the L2 cache size as there needs to be
sufficient cache space to hold the entire uncompressed extent, some of
the compressed data while it is uncompressed, and any additional
program state.

\item With more constrained disk bandwidth, gzip
%~\cite{GZIP} 
with
small extents is better than lzo as it trades a 10-30\% reduction in
decompression rate for a 10-40\% improvement in compression; if disk
bandwidth is a 
%limiter 
bottleneck then the improved compression is more valuable.
The additional compression from longer extents is minimal, and bzip2
while compressing much faster, decompresses so much slower that only a
highly imbalanced system would benefit from bzip2 decompression.

\item For online compression of data, lzf~\cite{LZF} with small to
moderate size extents is 
%the only choice.  Lzf compresses at roughly
best, as it compresses at roughly 100MB/s (about 10x
%100MB/s which is about 10$\times$ 
faster than gzip at level 1).
%Moderate sized extents may be a benefit if the data has unique strings
%as the unique packing option is very fast and gets very good
%compression.  Otherwise staying within the L2 cache remains a
%priority.
If the data has many unique strings, the unique packing option
is very fast and improves compression when combined with moderate
sized extents.
Otherwise, staying within the L2 cache remains a priority.
\end{itemize}

Access efficiency 
%turns out to be 
is more difficult to measure.  Our most
extensive analysis of this occurred using the 1998 World cup
traces~\cite{ita-wcweb98}.  These traces came with a sample analysis
program that calculated operations/server, minimum and maximum object
IDs, and other simple statistics.  Since the raw format of the data
was a binary structure, we expected that the DataSeries version would
have a faster wall clock time (because of parallel decompression), but
a slower CPU time.  We were surprised to learn that we actually used
less CPU time.  The reason 
%turned out to be 
was a combination of fread
inefficiency when used one record at a time, and a slow byte-swap
routine used by the sample program.  DataSeries avoids the former by
doing bulk processing and the latter by switching the ordering to the
native order when a file is written or repacked.  Further
investigation indicated that DataSeries was also executing unnecessary
instructions because our field accessors supported nullable fields,
but this happened to be unused in this format.  We added C++ template
based fields, but this did not reduce the penalty for access to the
data because the g++-3.3 compiler used with RHEL4 is fairly poor at
optimizing templated code.  However, with the g++-4.3 compiler, the
penalty is removed and the analysis code uses the same number of 
instructions on both sides.

% see src/paper/srtstat.cpp for data
This experience highlights the tradeoff between general and specific
code.  
%Above we saw the difference in performance that results from
%having fields that handle nullable and non-nullable data.  
Using the HP-UX block I/O traces,
we examined the performance difference for fields
that handle nullable and non-nullable data. 
%for the HP-UX block I/O traces.  
We
compared the same analysis using special case fields, templated
fields, basic fields, general fields, and general purpose program.  We
implemented special cases of the fields to work around the compiler
difficulties, and found that templated fields were 2\% slower than the
special case fields, the basic fields were another 1\% slower, and the
general fields were yet another 1\% slower.  The 
minor
performance differences
were because each field was used
exactly once, so most of the cost was in the memory access to the
field, which was usually not in cache as a different thread performed
decompression.  Conversely, the fully general program was
substantially slower (20\%) as it determines the expression at
runtime, and is capable of grouping on any field type while the
specific case programs can only group by 32-bit integer fields.  

We
expect that if we added prefetching code to the row analysis module we
would see a larger difference in the performance. 
We measured a
roughly 10x difference in access time through the basic and the
special case fields when using a program that repeatedly accessed the
same field and just accumulated the sum.  This is likely because of
the extra branch instructions interfering with the CPU's ability to
fully pipeline the code with the basic fields while the special case
fields reduce the access down to a single instruction which can be
combined with the add instruction.  The general fields on in cache
data are about 2.5x slower than the basic fields, showing the overhead
of the additional virtual function call.

%\subsection{Ellard Traces}\label{sec:ellard}
\subsection{A case study}\label{sec:ellard}

In an effort to experiment with using DataSeries to represent and
analyze traces generated by other people, we converted the NFS
traces used by Ellard \textit{et al.}~\cite{ellard03} into DataSeries.  
The ``Ellard traces''
were originally stored as compressed text files, one record per line.
The first part of each line is a series of fixed fields, followed by a
set of key-value pairs, and finally some debugging information.  
Their tools include a scanning program which reads the
trace files and outputs summary information for a trace.

Our evaluation consists of two parts.
%we wrote programs that
%converted between the two formats.  The reversible conversion
First, we wrote a reversible conversion program to
verify that we were properly preserving all of the information.
We found that the DataSeries files were on average 0.77x the size of
the original files when both were compressed using gzip.  The
compression improvements came as a result of the unique string
packing, delta encoding, and the elimination of the key fields through
the use of fixed field names and representing null fields in a single
bit before generic compression.

%Second, we
%wrote an analysis program that implemented the first three examples in
%the README that came with the Ellard tools.  
%These examples were
%all variants of a ``select count(*) group by field'' query.
%We found that our analysis
%program ran about 76x faster on those data files than the text
%analysis program that came with the distribution.  We also found that
%if we utilized lzo compression, which decompresses more quickly than
%gzip, our analysis program ran about 107x faster, in exchange for
%slightly larger (1.14x) data files.  This also illustrates how
%DataSeries can be optimized for a given purpose (e.g., greater
%compression for archival storage versus faster decompression for more
%efficient analysis). Our 
%experimental setup 
%is described in~\cite{DSTechnicalReportSnapshot}.

Second, we
wrote an analysis program that implemented the first three examples in
the README that came with the Ellard tools.  
These examples were
all variants of a ``select count(*) group by field'' query.
Table~\ref{tab:summary} presents our experimental results
(our experimental setup 
is described in~\cite{DSTechnicalReportSnapshot}).
The first two rows of Table~\ref{tab:summary} show the performance
of the Ellard tool.  The choice of compression algorithm (gzip or bzip2)
has only a minor effect in this case.
With gzip, ur analysis program ran about 76x faster on those data files 
(row 5).
We also found that
if we utilized lzo compression, which decompresses more quickly than
gzip, our analysis program ran about 107x faster (row 6), in exchange for
slightly larger (1.14x) data files.  This also illustrates how
DataSeries can be optimized for a given purpose (e.g., greater
compression for archival storage versus faster decompression for more
efficient analysis). 

The least speedup with our tool occurred with the bzip2 compression
algorithm and large extent size (row 9).
Although this configuration was still 20x faster than the Ellard tool,
it is noticeably slower than our other configurations.  The substantial
increase in system time for bzip2 (11.82s with 16MB extents versus
1.14s for lzo with 64KB extents) is a result of glibc's use of
mmap/munmap
for large allocations.  Every extent results in a separate pair of
mmap/munmap calls to the kernel, creating a substantial amount of page
zeroing in the kernel, and hence a large increase in the system time. 



% ORIGINAL TABLE 
%\begin{table*}
%\centering
%\begin{tabular}{|r|r|r|r|r|r|r|} \hline
%            & mean     & mean       & mean     & CPU     & mean     & Wall time \\
%algorithm   & user (s) & system (s) & CPU (s)  & speedup & wall (s) & speedup  \\ \hline
%ellard-gzip & 537.58    &  7.80     & 545.38   &  1.000x & 545.71   &   1.000x \\
%ellard-bzip2  & 638.48    & 12.68     & 651.16   &  0.836x & 571.49   &   0.955x \\
%\hline
%ds-gzip-512k  &  22.91    &  3.62     &  26.53   & 20.557x &   7.16   &  76.186x \\
%ds-gzip-64k   &  21.45    &  1.14     &  22.59   & 24.147x &   5.81   &  93.945x \\
%ds-gzip-128k  &  23.30    &  1.19     &  24.49   & 22.268x &   6.30   &  86.604x \\
%\hline
%ds-bzip2-16M  &  94.38    & 11.82     & 106.20   &  5.136x &  27.66   &  19.732x \\
%\hline
%ds-lzo-64k  &  18.71    &  1.14     &  19.85   & 27.472x &   5.10   & 106.897x \\
%ds-lzo-128k &  21.15    &  1.10     &  22.25   & 24.514x &   5.74   &  95.022x \\
%ds-lzo-512k &  24.07    &  4.07     &  28.14   & 19.382x &   7.40   &  73.762x \\ \hline
%\end{tabular}

%\caption{
%Summary of performance results for the two analysis programs
%operating on a variety of input files.  The analysis was run over the
%anon-home04-011118-* files.  For the ellard {\tt nfsscan} program
%the text files were compressed with either gzip or bz2.  For the
%DataSeries {\tt ellardanalysis} program, the DataSeries files were
%compressed with either gzip, bzip2, or lzo, and used various extent sizes
%as specified.  CPU and wall time are both relative to ellard-gzip.
%}

\begin{table*}
\centering
\begin{tabular}{|r||c|r|r|r|r|r|r|r|r|} \hline
 &  & \multicolumn{1}{c|}{compression} & \multicolumn{1}{c|}{extent}  & \multicolumn{3}{c|}{mean CPU time (s)}     & \multicolumn{1}{c|}{CPU}     & \multicolumn{1}{c|}{mean wall}    & \multicolumn{1}{c|}{wall time} \\ \cline{4-6}
row & tool & \multicolumn{1}{c|}{algorithm} & \multicolumn{1}{c|}{size}  & user & system & total  & \multicolumn{1}{c|}{speedup} & \multicolumn{1}{c|}{time (s)} & \multicolumn{1}{c|}{speedup}  \\ \hline
1 &Ellard & gzip &  & 537.58    &  7.80     & 545.38   &  1.000x & 545.71   &   1.000x \\
2 & Ellard & bzip2 & & 638.48    & 12.68     & 651.16   &  0.836x & 571.49   &   0.955x \\
\hline
3 & DataSeries & gzip & 64KB   &  21.45    &  1.14     &  22.59   & 24.147x &   5.81   &  93.945x \\
4 & DataSeries & gzip & 128KB  &  23.30    &  1.19     &  24.49   & 22.268x &   6.30   &  86.604x \\
5 & DataSeries & gzip & 512KB  &  22.91    &  3.62     &  26.53   & 20.557x &   7.16   &  76.186x \\
\hline
6 & DataSeries & lzo & 64KB  &  18.71    &  1.14     &  19.85   & 27.472x &   5.10   & 106.897x \\
7 & DataSeries & lzo &128KB &  21.15    &  1.10     &  22.25   & 24.514x &   5.74   &  95.022x \\
8 & DataSeries & lzo &512KB &  24.07    &  4.07     &  28.14   & 19.382x &   7.40   &  73.762x \\ \hline
9 & DataSeries & bzip2 &16MB  &  94.38    & 11.82     & 106.20   &  5.136x &  27.66   &  19.732x \\
\hline
\end{tabular}

\caption{
Detailed performance comparison for Ellard and DataSeries analysis programs.}


\label{tab:summary}
\end{table*}


%\fix{Jeff said ``What huh'' next to the beginning of this; dunno what to do about that.}

%Table~\ref{tab:summary} presents our experimental results, which
%show the
%significant speedup and reduction in CPU time 
%achieved 
%using DataSeries.  
%%The different sizes specified after the compression
%%algorithm for the DataSeries rows are the extent sizes. 
%%For example, ds-gz-512k corresponds to DataSeries with gzip
%%compression and a 512 KB extent size.
%Table~\ref{tab:summary} provides several interesting observations.
%For example, the substantial increase in system time 
%for bzip2 with large extents is a result of glibc's use of mmap/munmap
%for large allocations.  Every extent results in a separate pair of
%mmap/munmap calls to the kernel, hence a substantial amount of page
%zeroing in the kernel, and hence a large increase in the system time 
%used during the analysis run (e.g., 11.82s vs 1.14s of system time for 
%bz2 with 16MB extents versus lzo with 64KB extents).

\subsection{Other results}\label{sec:otherresults}

We have performed a number of other experiments that demonstrate
DataSeries' benefits. Due to space considerations, we will only
describe them briefly. One study of data compression showed that for a
relatively large disk I/O trace, DataSeries provides superior disk
space utilization; stored in DataSeries, the trace consumed 1.9GB of
disk space, but 14GB in CSV format and 8.5GB as a MySQL
database. Using the same trace in a series of analyses showed that
DataSeries delivers results 6-7x faster than both formats.

Our results and experience with DataSeries have also illustrated some
of the tradeoffs between generality and performance. We have found
that by adding various type-specific functionality, we can
significantly improve performance. For instance, for the analysis of
the World Cup traces, we added new accessors that did not check for
nullable fields, and found that we decreased CPU time by
8\%. Similarly, we found that we could ``group'' multiple analyses
into a single routine (therefore only needing to pass over the data
once), which can decrease the overall time taken by the same
factor. This is especially effective when compared to the time taken
in a more general solution (e.g., SQL queries) which cannot be combined
in this manner. The primary tradeoff is the added complexity, effort
and time and time involved in writing and using such type-specific
accessors, which may not be worth while for one-shot analyses.

% The lessons learned are scattered throughout the text, so this
% section was removed. 

%%%%%%% Beginning of removed section
\iffalse

\section{Lessons learned}\label{sec:lessonslearned}

\fix{this needs to become a summary section of highlights from lessons
learned. The list below are just ideas for this - all of them are discussed
in (varying) detail in the body of the paper}

Trick with cast to make data access fast: The format is designed for
efficient access. Values are packed so that once an extent is read in,
an analysis can iterate over the rows simply by increasing a single
counter, as if for an array of a structure.  Individual values are
accessed by an offset from that counter and a C++ cast.  applicable to
anyone storing similar data (RPC, streaming DB...)

Delta encoding via the pack\_relative field option saves space
with small cost.

Unique packing saves strings only once, improving on runlength
compression by having effectively an infinite window size for specific
strings.

Parallel compression and decompression improves performance
dramatically on multicore machines as shown by the 4x lower wall clock
time compared to CPU time on a 4 core machine in the ellard experiments.

Bulk processing of extents amortizes compression and decompression
costs over a fixed configurable amount of space.

Byte swapping if necessary is automatically performed.

Making data analyses type specific improves performance greatly.
Consider the analyses of NFS data.  When new accessors were added that
did not support nullable fields, performance improved by x amount.
Compared to SQL, and dsstatgroupby type specific performance can be as
much as x faster.  However, writing type specific accessors is more
time consuming, so for one shot analyses it may not be worth the
effort.

Discuss time field issues and why the field base value was a bad idea
and where we are with archival time representation and functional time
representation now.

field base value: applicable only to double fields,
this was used to gain precision for double values. This extra degree
of flexibility was introduced when we realized that storing nanosecond
precision time values, using the UNIX epoch (00:00:00 UTC on January
1, 1970), was not possible with a single double. Using this option, it
is possible to specific a base value that indicates the time the
events in the data file start, and the value stored indicates the time
in nanoseconds since.

\fi
%%%%%%% End of removed section


\section{Conclusions}\label{sec:conclusions}

We have described DataSeries, a data format that enables the efficient
and flexible storage and analysis
of structured serial data. This type of data is
used in numerous applications in all areas of computing and
science. We have identified the properties required for a system that
processes such data, and shown through a series of experiments and
comparisons to other systems that DataSeries satisfies these
properties. In particular, DataSeries offers significant performance
and storage efficiency benefits.  
% For these reasons, DataSeries is the
% recommended trace for the SNIA IOTTA technical working group.

We have also described the various lessons learned while designing and
implementing DataSeries. A general theme from this has been that, when
dealing with very large quantities of structured serial data, 
%with a particular workload,
there are significant performance, efficiency and reliability gains to be made
by giving up a relatively small amount of generality and flexibility. 
%
Overall, we believe that DataSeries should be considered for use in
any application that processes structured serial data. 
%DataSeries is
%open source, and available for download from
%http://tesla.hpl.hp.com/opensource/.

\bibliographystyle{abbrv}
{\small
\bibliography{tr-references}
}
% You must have a proper ".bib" file
%  and remember to run:
% latex bibtex latex latex
% to resolve all references
%
% ACM needs 'a single self-contained file'!
%
\balancecolumns

\end{document}
