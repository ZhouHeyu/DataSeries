\chapter{Introduction}\label{chap:introduction}

DataSeries\cite{dataseries-tr} is a trace file format and manipulation
library.  It is intended for storing structured serial data, so it is
similar to a append only SQL database\cite{wikipedia:sqldb} in that it
stores data organized into the equivalent of tables.  It is different
in that the rows of the tables have an order that is preserved.  It is
also different in that DataSeries does not at this point have a query
language interface.  To access DataSeries data, the programmer will
need to write a module to process the data, and compile and link that
module in to a program.  Dataseries also explicitly exposes the
underlying storage of the data in files, although this can be
partially hidden through the use of DataSeries indices.

This manual is divided into a number of sections:
\begin{itemize} 

\item Chapter~\ref{chap:programs} describes the general purpose
programs present as part of DataSeries.

\item Chapter~\ref{chap:analysis-module} describes how to write an
analysis module.

\item Chapter~\ref{chap:existing-modules} describes the existing
modules that can be used in DataSeries programs.

\item Chapter~\ref{chap:conversion-programs} describes how to write a
program that converts some existing format into DataSeries.

\item Chapter~\ref{chap:specific-analysis} describes the programs and 
modules that convert and analyze specific trace file formats.

\item Chapter~\ref{chap:file-format} described the DataSeries file
format.

\end{itemize}
