\chapter{Generic Programs}\label{chap:programs}

There are a number of existing programs present in DataSeries that can
be useful in working with DataSeries files.  We describe when one of
these programs would be used, and sketch the general usage of each
program here.  Detailed usage information on a program can be found on
that programs manual page.

Usually when you download DataSeries files from a remote repository,
the files will have been compressed with bz2\cite{BZIP2} compression.
This is done so that the files will be a small as possible during
transfer. However, bz2 is slow to decompress, and so it is useful to
convert the downloaded files to a faster decompression algorithm, and
sacrifice increased local storage space space.  The {\tt
dsrepack}\ref{program:dsrepack} program will perform this conversion.

The next most common operation is to take a quick look at the
downloaded DataSeries files.  The {\tt ds2txt}\ref{program:ds2txt}
program will take a DataSeries file and convert it to text so that the
type descriptions for the data stored in the file can be seen, and so
that a few of the sample lines can be examined.

A better understading of the data in the file may be gained by looking
at some statistics in the downloaded files.  This can be done by using
the {\tt dsstatgroupby}\ref{program:dsstatgroupby} program to
calculate statistics over various combinations of columns grouped by
another column.  This program can be looked at as a very restrictive
version of the SQL select statement.

Finally, before running a large number of analysis programs over the
downloaded files, it is usually useful to build an index over the
downloaded files.  This step can be performed by running {\tt
dsextentindex}\ref{program:dsextentindex}.  This generates an index
which can efficiently return the extents\ref{file-format:extent}
(sub-file chunks) that contain some amount of the desired data for the
analysis.

\section{{\tt dsrepack} -- re-compressing and merging DataSeries files}
\label{program:dsrepack}

\section{{\tt ds2txt} -- converting DataSeries files to text}
\label{program:ds2txt}

\section{{\tt dsstatgroupby} -- calculating statistics over input data}
\label{program:dsstatgroupby}

\section{{\tt dsextentindex} -- indexing DataSeries files}
\label{program:dsextentindex}


