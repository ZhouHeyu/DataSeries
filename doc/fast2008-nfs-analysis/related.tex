\section{Related work}
\label{sec:related}

The two closest pieces of related work are Ellard's 2001 NFS
study~\cite{EllardFast03}, and Leung's 2007 CIFS
study~\cite{LeungUsenix08}.  These papers also summarize the earlier
decade of filesystem tracing, so we refer interested readers to those
papers.  Ellard et al. captured NFS traces from a number of NetApp
filers on the Harvard campus, analyzed the traces and presented new
results looking at the sequentiality of the workload, and comparing
his results to earlier traces.  Ellard made his tools available, so we
initially considered building on top of them, but quickly discovered
that our workload was so much more intense that his tools would be
insufficient, and so we ended up building our own.  When we later
translated some of his traces to use our tools, we discovered our
tools were about 100x faster on a four core machine and used 25x less
CPU time for analysis.  Our 2003 traces were about 20x more intense
than Ellard's as we saw over 650 million operations/day, and Ellard
saw about 30.

Leung et al. traced a pair of NetApp filers on the NetApp campus.
Since the clients were entirely windows clients, his traces were of
CIFS data, and so he used the wireshark tools~\cite{wireshark} to
convert the traces.  Leung's traces were of comparable intensity to
Ellard's traces, and they noted that they had some small packet drops
during high load as they just used tcpdump for capture.  Leung found
complicated sequentiality patterns and provided an extensive analysis
of this.  Our 2007 traces were about 95x more intense than Leung's
traces as they saw a peak of 19.1 million operations/day and we saw an
average of about 1.8 billion.  This comparison is slightly misleading
as NFS tends to have more operations than CIFS because NFS is a
stateless protocol.

Tcpdump~\cite{tcpdump} is the tool that almost all papers use to capture packet
traces.  We initially tried using tcpdump, but experienced massive
packet loss using it in 2003, and so developed new techniques.  We
re-used the pcap file format that tcpdump uses for our raw captured
data.  When we captured our second set of traces in 2007, we needed to
capture at even higher rates, and so we ended up adopting a
specialized capture card, and wrote new capture software using
techniques we had developed in 2003 to allow us to capture above
5Gbits/s.

% capture.out a random capture from our nfs fileserver
% tcpdump -r capture.out| wc   
% 1,953,679 22682938 288138719
% ./nettrace2ds --convert --pcap --compress-lzf 0 160877 /tmp/foo.ds   4.74s user 0.45s system 124% cpu 4.159 total
% tshark -r /tmp/capture.out > /dev/null  44.25s user 0.30s system 99% cpu 44.564 total

Tcpdump also includes limited support for conversion of NFS packets.
The wireshark~\cite{wireshark} tool provides a graphical interface to
packet analysis, and the tshark tool provides conversion to text.  We
were not aware of wireshark at the time of our first capture, and we
simply adjusted our earlier tools when we did our 2007 tracing.  We
may consider using the wireshark converter in the future provided we
can determine how to make it run a lot faster.  Running tshark on a
sample 2 million packet capture took about 45 seconds whereas our
converter ran in about 5 seconds.  Given conversion takes 2-3 days for
a 5 day trace, we can't afford conversion to slow down by a factor of
$9\times$.

Some of the analysis techniques we use are derived from the database
community, namely the work on cubes~\cite{gray97cube} and approximate
quantiles~\cite{Manku98approximatemedians}.  We considered using a
standard SQL database for our storage and analysis, but abandoned that
quickly because a database that can hold 100 billion rows tends to be
very expensive.  We do use SQL databases for analysis and graphing
once we have reduced the data size down to a few million rows using
our tools.




