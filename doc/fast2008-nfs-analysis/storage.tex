\section{Storage}

Having decided to use a relational structuring for our data, we next
needed to decide how to store the data.  There were three primary
options available to us: text, SQL, and DataSeries, a specialized
format for storing trace data.  Text is a traditional way of storing
trace data, however, we were concerned that a text representation
would be too large and too slow.  Having later converted the Ellard
traces to DataSeries, we found that the DataSeries analysis used 25x
less CPU time~\cite{DSTechnicalReportSnapshot}.  This disparity
convinced us that text is an inappropriate format for storage of our
traces.

Since we chose a relational structuring for our data, an SQL database
would be a reasonable choice for storing our data.  However, most SQL
databases do not perform compression, and the few that do perform
relatively limited compression. Given that we were expecting to have
very large datasets, we did not feel that SQL would work well.  In
practice, we got about 6x compression (using gzip) from the minimal
size that an SQL database could store.  The generality of SQL
databases also means that we would expect a substantial penalty when
writing complex queries because we would have to extract all of the
data from the database, and they are not particularly fast at that
operation.

Therefore, we were going to need a specialized format that is more
efficient and compact for storing traces.  ....


