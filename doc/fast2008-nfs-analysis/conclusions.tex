\section{Conclusions}
\label{sec:conclusion}

We have described three improved techniques for packet capture on
networks.  The easily adopted technique should allow anyone capturing
NFS, CIFS, or iSCSI traffic from moderate performance storage systems
($\leq$1Gbit) to capture traffic with no losses.  The most advanced
technique allows lossless capture for 5-10Gbit storage systems, which
is at the high end of most file storage systems. 

We have provided guidelines for conversion, namely in terms of
parallelizing the conversion, retaining low level information, using
reversible anonymization when possible, and most importantly tagging
the trace data with version information to handle conversion bugs.

We have described our binary storage format that uses chunked
compression with multiple possible compression techniques, the use of
typed relational-style data structuring, the use of delta encoding,
and the use of type-safe, high-speed accessors.

We have explained the cube, and approximate quantile techniques that
we adopted from the database literature to handle analysis of very
large data sets.  We have also explained our hashtable, rotating
hash-map, and plotting techniques that we use for analyzing the data.

We have analyzed our NFS workload examining some of the different
properties found in a feature animation workload and demonstrating
that our techniques are effective.  Finally, we have made our traces
and tools available at http://anonymous.example.com

% Thank Alistair, Jay

