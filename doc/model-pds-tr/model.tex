\section{Performance model}
\label{sec:model}

This section presents a model for the runtime of a map-reduce job on a
hardware-efficient system.  It includes the model's assumptions,
parameters, and equations, along with a description of common
workloads.

\minorsection{Assumptions}
For a large class of data-intensive workloads, which we assume for our
model, computation time is negligible in comparison to I/O speeds.
Among others, this assumption holds for grep- and sort-like jobs, such
as those described by Dean and Ghemawat~\cite{mapreduce} as being
representative of most MapReduce jobs at Google, but may not hold in
other settings.
For workloads fitting the assumption, pipelined parallelism can allow
non-I/O operations to execute entirely in parallel with I/O operations,
such that overall throughput for each phase will be determined by the
I/O resource (network or storage) with the lowest effective throughput.
%In the absence of computationally expensive map or reduce
%tasks, we have observed that the I/O operators process data much
%slower than the CPU, so that the CPU is not the bottleneck.

For modeling purposes, we also do not consider specific network
topologies or technologies, and we assume that the network core is
over-provisioned enough that the internal network topology does not
impact the speeds of inter-node data transfers.  From our experience,
unlimited backplane bandwidth without any performance degradation is
probably impractical, although it was not an issue for our experiments
and we currently have no evidence for it causing issues on the other
large clusters which we analyze in Section~\ref{sec:discussion}.

The model assumes that input data is evenly distributed across all
participating nodes in the cluster, that nodes are homogeneous, and
that each node retrieves its initial input from local storage.  Most
map-reduce systems are designed to fit these assumptions.
\nocite{tashilocationaware} The model also accounts for output data
replication, assuming the common strategy of storing the first replica
on the local disks and sending the others over the network to other
nodes.  Finally, another important assumption is that a single job has
full access to the cluster at a time, with no competing jobs or other
activities.  Production map-reduce clusters may be shared by more than
one simultaneous job, but understanding a single job's performance is
a useful starting point.

%A common strategy that balances performance and reliability for 3-way
%replication is to store the first replica locally, the second replica
%on a different node of the same rack, and the third replica on a
%random node of a different rack.


\minorsection{Deriving the model from I/O operations}
Table~\ref{table:operations2} identifies the I/O operations in each
map-reduce phase for two variants of the sort operator.  When the data
fits in memory, a fast {\it in-memory sort} can be used. When it does
not fit, an {\it external sort} is used, which involves sorting each
batch of data in memory, writing it out to disk, and then reading and
merging the sorted batches into one sorted stream. The $\frac{n-1}{n}
d_m$ term appears in the equation, where $n$ is the number of nodes,
because in a well-balanced system each node partitions and transfers
that fraction of its mapped data over the network, keeping
$\frac{1}{n}$ of the data for itself.

{
\renewcommand{\baselinestretch}{1.0}
\begin{table}[t]
\centering
\begin{minipage}{1\textwidth}
\centering
\renewcommand{\arraystretch}{1.2}
\begin{tabular}{|l|l|l|}
\hline
        & $d_m < $ memory (in-memory sort) & $d_m \gg $ memory (external sort) \\ \hline
Phase 1
        & Disk read (input): $d_i$        & Disk read (input): $d_i$ \\
        & Disk write (backup): $d_m$      & Disk write (backup): $d_m$ \\
        & Network: $\frac{n-1}{n} d_m$    & Network: $\frac{n-1}{n} d_m$ \\
        &                                 & Disk write (sort): $d_m$ \\ \hline
Phase 2 & Network: $\left(r-1\right) d_o$ & Disk read (sort): $d_m$ \\
        &                                 & Network: $\left(r-1\right) d_o$ \\
        & Disk write (output): $r d_o$    & Disk write (output): $r d_o$ \\
\hline
\end{tabular}
\caption{I/O operations in a map-reduce job.
The first disk write in Phase 1 is an optional backup to protect
against failures.}
\label{table:operations2}
\end{minipage}
\end{table}
}


%NOTE: 4~GB sorts take up to 2.5 seconds on our cluster, so I had to
%take this out
%---on our four-core nodes, sorting 10 million 100-byte
%records takes only 0.3~seconds, which is a throughput of over 3~GB/s.

Table~\ref{table:symbols} lists the I/O speed and workload property
parameters of the model.
They include
%any replication factor used for the data output, and
%alternatives for describing the
amounts of data flowing through the system, which
%time, as altered by the map and reduce operations.  Those amounts can
%either be described
can be expressed either in absolute terms ($d_i$, $d_m$, and $d_o$) or in
terms of the ratios of the map and reduce operators' output and
input ($e_M$ and $e_R$, respectively).

{
\renewcommand{\baselinestretch}{1.0}
\begin{table}[t]
\centering
\begin{minipage}{1\textwidth}
\centering
\renewcommand{\arraystretch}{1.2}
\begin{tabular}{|l|p{13cm}|}
\hline
Symbol & Definition \\ \hline
$n$    & The number of nodes in the cluster. \\ \hline
$D_w$    & The aggregate disk \emph{write} throughput of a single node. A node with four disks, where each disk provides 65 MB/s writes, would have $D$ = 260
MB/s.
%\footnote{Disk speeds depend on the actual filesystem, the placement of blocks on the disks' tracks, and whether it is a write or a read.  
%Therefore, as discussed in Section~\ref{sec:optimality}, 
%we prefer to obtain these values empirically when possible. In our experiments, even when correcting for block placement to be in the fastest zone and using the faster \texttt{xfs} filesystem, write speeds lagged the raw device read throughput by 6-14\%.}
\\ \hline 
$D_r$    & The aggregate disk \emph{read} throughput of a single node. \\ \hline
$N$ & The network throughput of a single node.
% After accounting for unavoidable overhead, Gigabit Ethernet provides roughly $N$ = 110 MB/s.
\\ \hline 
$r$    & The replication factor used for the job's output data.  If no replication is used, $r=1$.  \\ \hline
$i$    & The total amount of input data for a given computation. \\ \hline
$d_i \left(= \frac{i}{n}\right)$  & The amount of input data per node, for a given computation. \\ \hline 
$d_m \left(= \frac{i \cdot e_M}{n}\right)$  & The amount of data per node after
the map operator, for a given computation. \\ \hline
$d_o \left(= \frac{i \cdot e_M \cdot e_R}{n}\right)$  & The amount of output
data per node, for a given computation. \\ \hline
$e_M \left(= \frac{d_m}{d_i}\right)$  & The ratio between the map operator's output and its input. \\ \hline
$e_R \left(= \frac{d_o}{d_m}\right)$  & The ratio between the reduce operator's output and its input.\\ \hline
\end{tabular}
\caption{Modeling parameters that include I/O speeds and workload properties.}
\label{table:symbols}
\end{minipage}
\end{table}
}


%Then, using the I/O operation patterns and modeling parameters, we derive our
Table~\ref{table:model:replication} gives the
model equations for the execution time of a map-reduce job
in each of four scenarios, representing the cross-product of the
Phase~1 backup write option ({\it yes} or {\it no}) and the sort type
({\it in-memory} or {\it external}).
In each case, the per-byte time to complete each phase (map and reduce)
is determined, summed, and multiplied by the number of input bytes per
node $\left(\frac{i}{n}\right)$.
The per-byte value for each phase is the larger (max) of that phase's
per-byte disk time and per-byte network time.
Using the last row (external sort, with backup write) as an example,
the map phase includes three disk transfers and one network transfer:
reading each input byte $\left(\frac{1}{D_r}\right)$, writing the $e_M$ map output
bytes to disk (the backup write; $\frac{e_M}{D_w}$),
writing $e_M$ bytes as part of the external sort $\left(\frac{e_M}{D_w}\right)$,
and sending $\frac{n-1}{n}$ of the $e_M$ map output bytes over the
network $\left(\frac{\frac{n-1}{n} e_M}{N}\right)$ to other reduce nodes.
The corresponding reduce phase includes two disk transfers and one
network transfer: reading sorted batches $\left(\frac{e_M}{D_r}\right)$,
writing $e_M e_R$ reduce output bytes produced locally $\left(\frac{e_M e_R}{D_w}\right)$
and $(r-1) e_M e_R$ bytes replicated from other nodes $\left(\frac{(r-1) e_M e_R}{D_w}\right)$,
and sending $e_M e_R$ bytes produced locally to $(r-1)$ other nodes
$\left(\frac{e_M e_R \left(r - 1\right)}{N}\right)$.
Putting all of this together produces the equation shown.


%{
\renewcommand{\baselinestretch}{1.0}
\begin{table*}
\centering
\renewcommand{\arraystretch}{1.2}
\begin{tabular}{|l|l|l|}
\hline
& $d_m < memory$ & $d_m \gg memory$ \\ \hline
Without backup write & $max\left\{\frac{d_i}{D_r},
\frac{\frac{n-1}{n} d_m}{N}\right\} + \frac{d_o}{D_w}$ & 
$max\left\{\frac{d_i}{D_r} + \frac{d_m}{D_w}, \frac{\frac{n-1}{n} d_m}{N}\right\} + \frac{d_m}{D_r} + \frac{d_o}{D_w}$ \\ \hline
With backup write & $max\left\{\frac{d_i}{D_r} + \frac{d_m}{D_w}, \frac{\frac{n-1}{n} d_m}{N}\right\} + \frac{d_o}{D_w}$ & $max\left\{\frac{d_i}{D_r} + \frac{2 d_m}{D_w},
\frac{\frac{n-1}{n} d_m}{N}\right\} + \frac{d_m}{D_r} + \frac{d_o}{D_w}$ \\ \hline
\end{tabular}
\caption{The execution time of a map-reduce computation on a parallel dataflow system.}
\label{table:model:simple}
\end{table*}
}

{
\renewcommand{\baselinestretch}{1.0}
\begin{table*}
\centering
\renewcommand{\arraystretch}{1.2}
\begin{tabular}{|l|l|l|}
\hline
& $d_m < memory$ (in-memory sort) \\ \hline
Without backup write &
$\frac{i}{n} \left( max\left\{\frac{1}{D_r}, \frac{\frac{n-1}{n} e_M}{N}\right\} +
max\left\{\frac{r e_M e_R}{D_w}, \frac{e_M e_R \left(r - 1\right)}{N}\right\} \right)$ \\ \hline
With backup write & $\frac{i}{n} \left( max\left\{\frac{1}{D_r} + \frac{e_M}{D_w},
\frac{\frac{n-1}{n} e_M}{N}\right\} + max\left\{\frac{r e_M e_R}{D_w}, \frac{e_M
e_R \left(r - 1\right)}{N}\right\} \right)$ \\ \hline

& $d_m \gg memory$ (external sort) \\ \hline
Without backup write & $\frac{i}{n} \left( max\left\{\frac{1}{D_r} + \frac{e_M}{D_w},
\frac{\frac{n-1}{n} e_M}{N}\right\} + max\left\{\frac{e_M}{D_r} + \frac{r e_M e_R}{D_w},
\frac{e_M e_R \left(r - 1\right)}{N}\right\} \right)$ \\ \hline
With backup write & $\frac{i}{n} \left(
max\left\{\frac{1}{D_r} + \frac{2 e_M}{D_w}, \frac{\frac{n-1}{n} e_M}{N}\right\} +
max\left\{\frac{e_M}{D_r} + \frac{r e_M e_R}{D_w}, \frac{e_M e_R \left(r -
1\right)}{N}\right\} \right)$ \\ \hline

\end{tabular}
\caption{Model equations for the execution time of a map-reduce computation on a parallel dataflow system.
%, as a function of the total amount of input data ($i$) and the map-reduce data expansion ratios ($e_M$ and $e_R$), where output data is replicated across $r$ nodes
}
\label{table:model:replication}
\end{table*}
}


\minorsection{Applying the model to common workloads}
Many workloads benefit from a parallel dataflow system because they run
on massive datasets, either extracting and processing a small amount of
interesting data or shuffling data from one representation to another.
%Data transfers
%can be on the order of Petabytes or Terabytes, as many blocks of data
%(each of 64 MB by default in Hadoop, but sometimes larger) are read
%from disk, processed, and transferred over the network.
We focus on parallel sort and grep in analyzing systems and validating
our model, which Dean and Ghemawat~\cite{mapreduce} indicate
are representative of most programs written by users of Google's
MapReduce.

For a grep-like job that selects a very small
fraction of the input data, $e_M \approx 0$ and $e_R = 1$, meaning
that only a negligible amount of data is (optionally) written to the
backup files, sent over the network, and written to the output files.
Thus, the best-case runtime is determined by the initial input disk
reads:

\begin{equation}
t_{\textstyle grep} = \frac{i}{n D_r}
\label{eqn:grepmodel}
\end{equation}

A sort workload maintains the same amount of data in both the map and
reduce phases, so $e_M = e_R = 1$.  If the amount of data
per node is small enough to accommodate an in-memory sort and not
warrant a Phase~1 backup, the top equation of
Table~\ref{table:model:replication} is used, simplifying to:

\begin{equation}
t_{\textstyle sort} = \frac{i}{n}
\left( max\left\{\frac{1}{D_r}, \frac{n-1}{n N}\right\}
+ max\left\{\frac{r}{D_w}, \frac{r-1}{N}\right\} \right)
\label{eqn:sortmodel2}
\end{equation}


\minorsection{Determining input parameters for the model}
Appropriate parameter values are a crucial aspect of model accuracy,
whether using the model to evaluate how well a production system is
performing or to determine what should be expected from a hypothetical system.
%\minorsection{Configuration parameters}
The $n$ and $r$ parameters are system configuration choices that can
be applied directly in the model for both production and hypothetical
systems.

%\minorsection{Workload property parameters}
The amount of data flowing through various operators ($d_i$,
$d_m$, or $d_o$) depend upon the characteristics of the map and
reduce operators and of the data itself.  For a production system,
they can be measured and then plugged into a model that evaluates the
performance of a given workload run on that system.  For a
hypothetical system, or if actual system measurements are not
available, some estimates must be used, such as $d_i = d_m = d_o$ for
sort or $d_m = d_o = 0$ for grep.

The determination of which equation to use, based on the backup write
option and sort type choices, is also largely dependent on the
workload characteristics, but in combination with system
characteristics.  Specifically, the sort type choice depends on the
relationship between $d_m$ and the amount of main memory available for
the sort operator.  The backup write option is a softer choice, worthy
of further study, involving the time to do a backup write
$\left(\frac{d_m}{D_w}\right)$, the total execution time of the job,
and the likelihood of a node failure during the job's execution.  Both
Hadoop and Google's MapReduce always do the backup write, at least to
the local file system cache.

%\minorsection{I/O speed parameters}
The appropriate values for I/O speed depend on what is being evaluated.
For both production and hypothetical systems, specification values
for the hardware can be used---for example, 1~Gbps for the network
and the maximum streaming bandwidth specified for the given disk(s).
This approach is appropriate for evaluating the efficiency of the
entire software stack, from the operating system up.
However, if the focus is on the programming framework, using raw
hardware specifications can indicate greater inefficiency than is
actually present.
In particular, some efficiency is generally lost in the underlying
operating system's conversion of raw disk and network resources into
higher level abstractions, such as file systems and network sockets.
To focus attention on programming framework inefficiencies,
one should use measurements of the disk and network bandwidths available
to applications using the abstractions.
As shown in our experiments, such measured values are lower than
specified values and often have non-trivial characteristics, such as
dependence on file system age or network communication patterns.
%In our evaluations of existing systems, we discuss both approaches.

